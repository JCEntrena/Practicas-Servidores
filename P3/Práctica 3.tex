%%%%
% Modificación de una plantilla de Latex para adaptarla al castellano.
%%%

%%%%%%%%%%%%%%%%%%%%%%%%%%%%%%%%%%%%%%%%%
% Thin Sectioned Essay
% LaTeX Template
% Version 1.0 (3/8/13)
%
% This template has been downloaded from:
% http://www.LaTeXTemplates.com
%
% Original Author:
% Nicolas Diaz (nsdiaz@uc.cl) with extensive modifications by:
% Vel (vel@latextemplates.com)
%
% License:
% CC BY-NC-SA 3.0 (http://creativecommons.org/licenses/by-nc-sa/3.0/)
%
%%%%%%%%%%%%%%%%%%%%%%%%%%%%%%%%%%%%%%%%%

%----------------------------------------------------------------------------------------
%	PACKAGES AND OTHER DOCUMENT CONFIGURATIONS
%----------------------------------------------------------------------------------------

\documentclass[a4paper, 11pt]{article} % Font size (can be 10pt, 11pt or 12pt) and paper size (remove a4paper for US letter paper)

\usepackage[protrusion=true,expansion=true]{microtype} % Better typography
\usepackage{graphicx} % Required for including pictures
\usepackage[usenames,dvipsnames]{color} % Coloring code
\usepackage{wrapfig} % Allows in-line images
\usepackage[utf8]{inputenc}
\usepackage{enumerate}
\usepackage{enumitem}

% Imágenes
\usepackage{graphicx} 

\usepackage{amsmath}
% para importar svg
%\usepackage[generate=all]{svgfig}

% sudo apt-get install texlive-lang-spanish
\usepackage[spanish]{babel} % English language/hyphenation
\selectlanguage{spanish}
% Hay que pelearse con babel-spanish para el alineamiento del punto decimal
\decimalpoint
\usepackage{dcolumn}
\newcolumntype{d}[1]{D{.}{\esperiod}{#1}}
\makeatletter
\addto\shorthandsspanish{\let\esperiod\es@period@code}
\makeatother

\usepackage{longtable}
\usepackage{tabu}
\usepackage{supertabular}

\usepackage{multicol}
\newsavebox\ltmcbox

% Símbolos matemáticos
\usepackage{amssymb}
\let\oldemptyset\emptyset
\let\emptyset\varnothing

%URL's en comentarios
\usepackage{url}

\usepackage[section]{placeins} % Para gráficas en su sección.
\usepackage{mathpazo} % Use the Palatino font
\usepackage[T1]{fontenc} % Required for accented characters
\newenvironment{allintypewriter}{\ttfamily}{\par}
\setlength{\parindent}{0pt}
\parskip=8pt
\linespread{1.05} % Change line spacing here, Palatino benefits from a slight increase by default

\makeatletter
\renewcommand\@biblabel[1]{\textbf{#1.}} % Change the square brackets for each bibliography item from '[1]' to '1.'
\renewcommand{\@listI}{\itemsep=0pt} % Reduce the space between items in the itemize and enumerate environments and the bibliography
\newcommand{\imagen}[2]{\begin{center} \includegraphics[width=90mm]{#1} \\#2 \end{center}}

\renewcommand{\maketitle}{ % Customize the title - do not edit title and author name here, see the TITLE block below
\begin{flushright} % Right align
{\LARGE\@title} % Increase the font size of the title

\vspace{60pt} % Some vertical space between the title and author name

{\large\@author} % Author name
\\\@date % Date

\vspace{40pt} % Some vertical space between the author block and abstract
\end{flushright}
}
\usepackage[hidelinks]{hyperref}

  % Para las enumeraciones anidadas y sus referencias, basado en http://stackoverflow.com/questions/691351/how-to-customize-references-to-sublists-in-latex
  \renewcommand{\theenumi}{\arabic{enumi}.}
  \renewcommand{\theenumii}{\arabic{enumii}}
  \renewcommand{\theenumiii}{\arabic{enumiii}}
  
  \renewcommand{\labelenumi}{\theenumi}
  \renewcommand{\labelenumii}{\theenumi\theenumii.}
  \renewcommand{\labelenumiii}{\theenumi\theenumii.\theenumiii.}
  
  \makeatletter
  \renewcommand{\p@enumii}{\theenumi}
  \renewcommand{\p@enumiii}{\theenumi\theenumii.}
%----------------------------------------------------------------------------------------
%	TITLE
%----------------------------------------------------------------------------------------

\title{\textbf{Práctica 3}\\ % Title
} % Subtitle

\author{\textsc{José Carlos Entrena} % Author
\\{\textit{Universidad de Granada}}} % Institution

\date{\today} % Date

%----------------------------------------------------------------------------------------

\begin{document}

\maketitle % Imprime el título.

{\parskip=2pt
  \tableofcontents
}   % Índice.

\pagebreak % Cierra la página.

\section{Resolución de los ejercicios}

\subsection{Ejercicio 1}
a) ¿En qué archivos se guarda registro de los paquetes instalados en sistema con los gestores de paquetes de Ubuntu y CentOS? Durante la práctica 2 instaló LAMP como un único paquete o instalando cada componente diferenciado. Busque en el archivo de registro las líneas correspondienter a la instalación y preséntelas. 

b) En el directorio /var/log es común encontrar archivos con extensiones en formato <número>.gz. Por ejemplo, .1.gz. Explique cómo se generan estos archivos y qué relación guardan entre ellos. 


\subsection{Ejercicio 2}
Indique los pasos que ha seguido, comandos empleados y significado de los mismo. Junto a cada comando, presente las líneas del registro que son significativas en cada paso: indicación de fallo, reemplazo, inicio y finalización de la reconstrucción del RAID. 

\subsection{Ejercicio 3}
Añada a al configuración de cron una tarea que se ejecute diariamente y que copie una vez al día el contenido del directorio ~/codigo a ~/seguridad/\$fecha donde \$fecha es la fecha actual del sistema (puede usar el comando date). Otra tarea, ejecutará una vez al mes y reunirá todos los directorios diarios creados para el mes pasado en un archivo ~/seguridad/dirCodigo.<número>.gz. Presente las líneas de configuración de cron afectadas, explicando su significado. Si crea ficheros por lotes, presente y explique el código. 

\subsection{Ejercicio 4}
a) Pruebe a ejecutar el comando, conectar un dispositivo USB y vuelva a ejecutar el comando. Copie y pegue las líneas que hacen mención al dispositivo conectado (considere usar dmesg | tail). 

b) Explique las diferencias (si las hay) entre o consulaar el contenido de un archivo /var/log/dmesg?\footnote{Supongo un error en la redacción de la pregunta, y se lo notifico aquí para futuras correciones.}

\subsection{Ejercicio 5}
a) Ejecute el monitor de "System Performance" y muestre el resultado. Incluya capturas de pantalla y comente la información que aparece. 

b) Cree un recopilador de datos definido por el usuario (modo avanzado) que incluya tanto el contador de rendimiento como los datos de seguimiento: 
\begin{itemize}
\item Todos los referentes al procesador, al proceso y al servicio web. 
\item Intervalo de muestra 15 segundos.
\item Almacene el resultado en el directorio Escritorio/logs
\end{itemize}
Incluya las capturas de pantalla de cada paso. 

\subsection{Ejercicio 6}
Elija uno de los sistemas de monitorización descritos en este apartado e instálelo. Describa los pasas seguidos así como posibles incidencias en la instalación que ha debido resolver. Monitorice uno o varios de sus servidores y presente ejemplos de algunas medidas que considere significativas, explicando su significado y los valores reales observados. 

\subsection{Ejercicio 7}
Diseñe un pequeño modelo de BBDDs para un problema de su elección e impleméntelo en MySQL. También puede emplear un sistema de código abierto del que conozca su diseño de BBDDs. Plantee una combinación de consultas (al menos dos) que considere significativas y explique los resultados obtenidos en su "profile". Se valorará especialmente que sea capaz de introducir cambios en el diseño de tablas o en las consultas que mejore los resultados y que sepa justificar la mejora. 

\end{document}