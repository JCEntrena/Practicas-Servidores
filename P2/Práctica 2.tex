%%%%
% Modificación de una plantilla de Latex para adaptarla al castellano.
%%%

%%%%%%%%%%%%%%%%%%%%%%%%%%%%%%%%%%%%%%%%%
% Thin Sectioned Essay
% LaTeX Template
% Version 1.0 (3/8/13)
%
% This template has been downloaded from:
% http://www.LaTeXTemplates.com
%
% Original Author:
% Nicolas Diaz (nsdiaz@uc.cl) with extensive modifications by:
% Vel (vel@latextemplates.com)
%
% License:
% CC BY-NC-SA 3.0 (http://creativecommons.org/licenses/by-nc-sa/3.0/)
%
%%%%%%%%%%%%%%%%%%%%%%%%%%%%%%%%%%%%%%%%%

%----------------------------------------------------------------------------------------
%	PACKAGES AND OTHER DOCUMENT CONFIGURATIONS
%----------------------------------------------------------------------------------------

\documentclass[a4paper, 11pt]{article} % Font size (can be 10pt, 11pt or 12pt) and paper size (remove a4paper for US letter paper)

\usepackage[protrusion=true,expansion=true]{microtype} % Better typography
\usepackage{graphicx} % Required for including pictures
\usepackage[usenames,dvipsnames]{color} % Coloring code
\usepackage{wrapfig} % Allows in-line images
\usepackage[utf8]{inputenc}
\usepackage{enumerate}
\usepackage{enumitem}

% Imágenes
\usepackage{graphicx} 

\usepackage{amsmath}
% para importar svg
%\usepackage[generate=all]{svgfig}

% sudo apt-get install texlive-lang-spanish
\usepackage[spanish]{babel} % English language/hyphenation
\selectlanguage{spanish}
% Hay que pelearse con babel-spanish para el alineamiento del punto decimal
\decimalpoint
\usepackage{dcolumn}
\newcolumntype{d}[1]{D{.}{\esperiod}{#1}}
\makeatletter
\addto\shorthandsspanish{\let\esperiod\es@period@code}
\makeatother

\usepackage{longtable}
\usepackage{tabu}
\usepackage{supertabular}

\usepackage{multicol}
\newsavebox\ltmcbox

% Símbolos matemáticos
\usepackage{amssymb}
\let\oldemptyset\emptyset
\let\emptyset\varnothing

%URL's en comentarios
\usepackage{url}

\usepackage[section]{placeins} % Para gráficas en su sección.
\usepackage{mathpazo} % Use the Palatino font
\usepackage[T1]{fontenc} % Required for accented characters
\newenvironment{allintypewriter}{\ttfamily}{\par}
\setlength{\parindent}{0pt}
\parskip=8pt
\linespread{1.05} % Change line spacing here, Palatino benefits from a slight increase by default

\makeatletter
\renewcommand\@biblabel[1]{\textbf{#1.}} % Change the square brackets for each bibliography item from '[1]' to '1.'
\renewcommand{\@listI}{\itemsep=0pt} % Reduce the space between items in the itemize and enumerate environments and the bibliography
\newcommand{\imagen}[2]{\begin{center} \includegraphics[width=90mm]{#1} \\#2 \end{center}}

\renewcommand{\maketitle}{ % Customize the title - do not edit title and author name here, see the TITLE block below
\begin{flushright} % Right align
{\LARGE\@title} % Increase the font size of the title

\vspace{60pt} % Some vertical space between the title and author name

{\large\@author} % Author name
\\\@date % Date

\vspace{40pt} % Some vertical space between the author block and abstract
\end{flushright}
}
\usepackage[hidelinks]{hyperref}

  % Para las enumeraciones anidadas y sus referencias, basado en http://stackoverflow.com/questions/691351/how-to-customize-references-to-sublists-in-latex
  \renewcommand{\theenumi}{\arabic{enumi}.}
  \renewcommand{\theenumii}{\arabic{enumii}}
  \renewcommand{\theenumiii}{\arabic{enumiii}}
  
  \renewcommand{\labelenumi}{\theenumi}
  \renewcommand{\labelenumii}{\theenumi\theenumii.}
  \renewcommand{\labelenumiii}{\theenumi\theenumii.\theenumiii.}
  
  \makeatletter
  \renewcommand{\p@enumii}{\theenumi}
  \renewcommand{\p@enumiii}{\theenumi\theenumii.}
%----------------------------------------------------------------------------------------
%	TITLE
%----------------------------------------------------------------------------------------

\title{\textbf{Práctica 1}\\ % Title
} % Subtitle

\author{\textsc{José Carlos Entrena} % Author
\\{\textit{Universidad de Granada}}} % Institution

\date{\today} % Date

%----------------------------------------------------------------------------------------

\begin{document}

\maketitle % Imprime el título.

{\parskip=2pt
  \tableofcontents
}   % Índice.

\pagebreak % Cierra la página.

\section{Resolución de los ejercicios}


\subsection{Ejercicio 1}
Proporcione ejemplos de llamada a yum para buscar, instalar y eliminar
paquetes (Pista: man yum)

Para buscar paquetes utilizaremos la orden yum search 'paquete', aunque podemos usar yum list si conocemos el nombre completo.

Si queremos instalar un paquete, utilizaremos la orden yum install 'paquete', o yum reinstall, para reinstalar uno que ya tenemos. 

Finalmente, si lo que queremos es eliminar un paquete, requeriremos de la orden  yum remove 'paquete', o yum erase. Nótese que también se eliminan los paquetes dependientes del que eliminamos, y que existen ciertos paquetes protegidos contra el borrado (el propio yum, por ejemplo).\footnote{Referencias: man yum}

\subsection{Ejercicio 2}
¿Qué ha de hacer para que yum pueda tener acceso a Internet a través
de un proxy?(Pistas: archivo de configuración en /etc, proxy: stargate.ugr.es:3128).
¿Cómo añadimos un nuevo repositorio? 

En el archivo de configuración /etc/yum.conf hemos de añadir la línea: 
proxy = http://proxy-host:proxy-port

Para la autenticación, en caso de ser necesaria, tendremos las líneas:\\
proxy-username=proxy-username\\
proxy-password=proxy-password \footnote{\url{https://docs.openathens.net/display/public/OALA221/Configuring+yum+to+use+a+proxy+server;jsessionid=1ADC56498F0FF545643F622A5322B447}}

Para añadir un repositorio, tenemos varias formas: 

	La primera de ellas es a través del archivo /etc/yum.conf, añadiendo una sección [repository] y dentro de ella los repositorios de la forma: 

	[repository ID]\\
	name=repo-name\\
	baseurl=url, archivo o ftp://repo-path\footnote{\url{https://www.centos.org/docs/5/html/5.2/Deployment_Guide/s1-yum-yumconf-repository.html}}
	
	La segunda forma es añadirlo mediante línea de comandos, ejecutando la siguiente órden como root: 
	
	yum-config-manager --add-repo repo-url

	La tercera y última forma es añadir el repositorio en formato .repo al archivo /etc/yum.repos.d, de la misma forma que en /etc/yum.conf 
	
	Para utilizar este repositorio deberemos activarlo, o bien estableciendo el parámetro enabled=1 en el archivo de configuración correspondiente, o desde línea de comandos, con la orden (en modo root):
	
	yum-config-manager --enable repo-name\footnote{\url{https://access.redhat.com/documentation/en-US/Red_Hat_Enterprise_Linux/6/html/Deployment_Guide/sec-Managing_Yum_Repositories.html}} 
	

\subsection{Ejercicio 3}
Proporcione ejemplos de comandos para buscar un paquete en un
repositorio y el correspondiente para instalarlo. (Pista: man apt-get ; man apt-cache)

Para instalar un paquete en Ubuntu Server hemos de utilizar el comando: 

"sudo apt-get install 'paquete'" 

mientras que para buscar un paquete usaremos 

"apt-cache search 'paquete'" 

Por ejemplo, para buscar un programa de ajedrez, ejecutaremos sudo apt-cache search chess, y para instalar el paquete chessx, sudo apt-get install chessx.\footnote{Referencias: man apt-get, man apt-cache}

\subsection{Ejercicio 4}
Indiqué como debe modificar la configuración de apt para acceder a los repositorios a través del proxy. ¿Cómo añadimos un nuevo repositorio?

Primero configuramos la dirección http con el comando: 

export http\_proxy=http://<proxy>:<port>   o \\
export https\_proxy=http://<proxy>:<port>

después ejecutamos la orden: 

sudo -E add-apt-repository repositorio

y en caso de ser necesario un login, usamos: 

export https\_proxy=<username>:<password>@<proxy>:<port>\footnote{\url{http://askubuntu.com/questions/53146/how-do-i-get-add-apt-repository-to-work-through-a-proxy}}

\subsection{Ejercicio 5}
¿Qué diferencia hay entre telnet y ssh?

SSH cifra toda la conexión proporcionando un buen nivel de seguridad, mientras que telnet no cifra nada, enviando la información en texto plano.


\subsection{Ejercicio 6}
Modifique la configuración de SSH para que impida el acceso remoto del usuario root y cambie el puerto por defecto. Indique las líneas modificadas en el fichero de configuración y ponga de manifiesto el cambio mediante capturas de pantallas en las que se aprecie el comportamiento antes y después de los cambios. Tenga en cuenta que debe reiniciar el servicio para que tome los cambios.

Tendremos que modificar el archivo de configuración /etc/ssh/sshd\_config para poder hacer estos cambios. Para modificar el puerto por defecto solo tenemos que modificar el parámetro puerto en el archivo, especificando el nuevo puerto que queremos usar, y para permitir que el root no pueda loguearse, cambiaremos la línea PermitRootLogin para ajustarla a "no".\footnote{\url{https://www.digitalocean.com/community/tutorials/how-to-use-ssh-to-connect-to-a-remote-server-in-ubuntu}}

Podemos ver los cambios en las figuras 1 y 2. 

\pagebreak

\begin{figure}[htpb]
\includegraphics[height=11cm]{../../../../root}
\caption{Cambio de puerto y denegación a root.}
\end{figure}

\pagebreak 

\begin{figure}[htpb]
\includegraphics[height=11cm]{../../../../login}
\caption{Correcto funcionamiento con mi usuario.}
\end{figure}


\subsection{Ejercicio 7}
Configure una instancia de Linux de forma que pueda acceder remotamente (desde otra instancia o desde su anfitrión) sin introducir contraseña (Pistas: ssh-keygen, ssh-copy-id). Documente el proceso que ha seguido indicando y explicando los comandos utilizados así como posibles cambios en la configuración del servicio. Muestre con capturas de pantalla que puede conectar al servidor ssh remoto sin introducir contraseña.


\subsection{Ejercicio 8}
En muchas ocasiones es necesario reiniciar un servicio para que tome los cambios en su configuración. Indique los comandos que puede emplear en Ubuntu y CentOS para hacerlo.

Aunque depende del servicio en cuestión, la forma más usual de reiniciar un servicio es la siguiente: 

En CentOS, poniendo como ejemplo a MySQL; el comando a ejecutar sería: sudo /etc/init.d/mysqld restart'. 

En Ubuntu, poniendo como ejemplo a apache, deberíamos ejecutar el comando: 'sudo /etc/init.d/apache2 restart'. Para ssh, que lo hemos usado anteriormente, podemos usar 'sudo restart ssh'. 

En general, iremos a donde el servicio esté alojado y ejecutaremos el comando asignado a cada uno para reiniciarlo. 

\subsection{Ejercicio 9}
Ponga de manifiesto el funcionamiento de PHP en Apache creando un fichero php que presente su nombre y apellidos y accediéndolo con un navegador web. Presente la captura de pantalla del resultado. Ponga de manifiesto el funcionamiento de MySQL accediendo a la BBDDs por defecto (mysql) y consultando los usuarios definidos en el sistema (select * from user). Documente con capturas de pantalla el acceso y resultado de la consulta.



MySQL, en Ubuntu Server. 

Iniciamos MySQL con el comando mysql -u root -p, y una vez dentro, elegimos la base de datos por defecto con use mysql, y ejecutamos la sentencia dada. En la siguiente figura podemos ver un extracto del resultado.\footnote{Me ha sido imposible copiar la consulta completa, pues Ubuntu Server no me deja ver más que lo que se muestra en pantalla, no pudiendo subir a ver el resto.}

\begin{figure}[htpb]
\includegraphics[height=11cm]{../../../../sql}
\caption{Consulta a la base de datos. }
\end{figure}


\subsection{Ejercicio 10}
Para poner de manifiesto que el servidor está funcionando, acceda con un navegador web a su propio equipo (localhost). Cree una página HTML básica con su nombre y apellidos y publíquela en su servidor IIS. Muestre, con una captura de pantalla, como accede a dicha página con el navegador web.


\subsection{Ejercicio 11}
Escriba un breve contenido en un fichero de texto plano, cópielo y modifíquelo ligeramente en un segundo archivo, por ejemplo, añadiendo un par de líneas. Calcule las diferencias entre el fichero original y el modificado. Indique los comandos necesarios para aplicar el parche así generado sobre el primer archivo y obtener el segundo. Documente el proceso con capturas de pantalla de cada paso.

Ejercicio hecho en Ubuntu Server. 

He creado dos archivos de texto de ejemplo, llamados nuevo y nuevo2, donde nuevo2 añade y quita algún texto de nuevo. Para crear el archivo patch que tenga la diferencia de los dos, usamos el comando diff -u nuevo nuevo2 > n.patch, guardando los cambios en un archivo .patch. Podemos ver el resultado en la Figura 3. 

\begin{figure}[htpb]
\includegraphics[height=11cm]{../../../../diff}
\caption{Creación del patch.}
\end{figure}

\pagebreak

Una vez creado el patch, queremos aplicarlo a nuevo para obtener nuevo2. Esto lo haremos con el comando patch, mediante la orden\\ patch nuevo <\ n.patch. Podemos ver el resultado en la siguiente figura. \footnote{\url{http://jungels.net/articles/diff-patch-ten-minutes.html}}

\begin{figure}[htpb]
\includegraphics[height=11cm]{../../../../patch}
\caption{Aplicación del patch.}
\end{figure}


\subsection{Ejercicio 12}
Realice la instalación de esta aplicación y pruebe a modificar algún parámetro de algún servicio. Muestre las capturas de el proceso de modificación y ponga de manifiesto el resultado.




\end{document}