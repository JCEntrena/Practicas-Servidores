%%%%
% Modificación de una plantilla de Latex para adaptarla al castellano.
%%%

%%%%%%%%%%%%%%%%%%%%%%%%%%%%%%%%%%%%%%%%%
% Thin Sectioned Essay
% LaTeX Template
% Version 1.0 (3/8/13)
%
% This template has been downloaded from:
% http://www.LaTeXTemplates.com
%
% Original Author:
% Nicolas Diaz (nsdiaz@uc.cl) with extensive modifications by:
% Vel (vel@latextemplates.com)
%
% License:
% CC BY-NC-SA 3.0 (http://creativecommons.org/licenses/by-nc-sa/3.0/)
%
%%%%%%%%%%%%%%%%%%%%%%%%%%%%%%%%%%%%%%%%%

%----------------------------------------------------------------------------------------
%	PACKAGES AND OTHER DOCUMENT CONFIGURATIONS
%----------------------------------------------------------------------------------------

\documentclass[a4paper, 11pt]{article} % Font size (can be 10pt, 11pt or 12pt) and paper size (remove a4paper for US letter paper)

\usepackage[protrusion=true,expansion=true]{microtype} % Better typography
\usepackage{graphicx} % Required for including pictures
\usepackage[usenames,dvipsnames]{color} % Coloring code
\usepackage{wrapfig} % Allows in-line images
\usepackage[utf8]{inputenc}
\usepackage{enumerate}
\usepackage{enumitem}

% Imágenes
\usepackage{graphicx} 

\usepackage{amsmath}
% para importar svg
%\usepackage[generate=all]{svgfig}

% sudo apt-get install texlive-lang-spanish
\usepackage[spanish]{babel} % English language/hyphenation
\selectlanguage{spanish}
% Hay que pelearse con babel-spanish para el alineamiento del punto decimal
\decimalpoint
\usepackage{dcolumn}
\newcolumntype{d}[1]{D{.}{\esperiod}{#1}}
\makeatletter
\addto\shorthandsspanish{\let\esperiod\es@period@code}
\makeatother

\usepackage{longtable}
\usepackage{tabu}
\usepackage{supertabular}

\usepackage{multicol}
\newsavebox\ltmcbox

% Símbolos matemáticos
\usepackage{amssymb}
\let\oldemptyset\emptyset
\let\emptyset\varnothing

%URL's en comentarios
\usepackage{url}

\usepackage[section]{placeins} % Para gráficas en su sección.
\usepackage{mathpazo} % Use the Palatino font
\usepackage[T1]{fontenc} % Required for accented characters
\newenvironment{allintypewriter}{\ttfamily}{\par}
\setlength{\parindent}{0pt}
\parskip=8pt
\linespread{1.05} % Change line spacing here, Palatino benefits from a slight increase by default

\makeatletter
\renewcommand\@biblabel[1]{\textbf{#1.}} % Change the square brackets for each bibliography item from '[1]' to '1.'
\renewcommand{\@listI}{\itemsep=0pt} % Reduce the space between items in the itemize and enumerate environments and the bibliography
\newcommand{\imagen}[2]{\begin{center} \includegraphics[width=90mm]{#1} \\#2 \end{center}}

\renewcommand{\maketitle}{ % Customize the title - do not edit title and author name here, see the TITLE block below
\begin{flushright} % Right align
{\LARGE\@title} % Increase the font size of the title

\vspace{60pt} % Some vertical space between the title and author name

{\large\@author} % Author name
\\\@date % Date

\vspace{40pt} % Some vertical space between the author block and abstract
\end{flushright}
}
\usepackage[hidelinks]{hyperref}

  % Para las enumeraciones anidadas y sus referencias, basado en http://stackoverflow.com/questions/691351/how-to-customize-references-to-sublists-in-latex
  \renewcommand{\theenumi}{\arabic{enumi}.}
  \renewcommand{\theenumii}{\arabic{enumii}}
  \renewcommand{\theenumiii}{\arabic{enumiii}}
  
  \renewcommand{\labelenumi}{\theenumi}
  \renewcommand{\labelenumii}{\theenumi\theenumii.}
  \renewcommand{\labelenumiii}{\theenumi\theenumii.\theenumiii.}
  
  \makeatletter
  \renewcommand{\p@enumii}{\theenumi}
  \renewcommand{\p@enumiii}{\theenumi\theenumii.}
%----------------------------------------------------------------------------------------
%	TITLE
%----------------------------------------------------------------------------------------

\title{\textbf{Práctica 1}\\ % Title
} % Subtitle

\author{\textsc{José Carlos Entrena} % Author
\\{\textit{Universidad de Granada}}} % Institution

\date{\today} % Date

%----------------------------------------------------------------------------------------

\begin{document}

\maketitle % Imprime el título.

{\parskip=2pt
  \tableofcontents
}   % Índice.

\pagebreak % Cierra la página.

\section{Resolución de los ejercicios}

\subsection{Ejercicio 1}
Proporcione ejemplos de llamada a a yum para buscar, instalar y eliminar
paquetes (Pista: man yum)


\subsection{Ejercicio 2}
¿Qué ha de hacer para que yum pueda tener acceso a Internet a través
de un proxy?(Pistas: archivo de configuración en /etc, proxy: stargate.ugr.es:3128).
¿Cómo añadimos un nuevo repositorio?


\subsection{Ejercicio 3}
Proporcione ejemplos de comandos para buscar un paquete en un
repositorio y el correspondiente para instalarlo. (Pista: man apt-get ; man apt-cache)


\subsection{Ejercicio 4}
Indiqué como debe modificar la configuración de apt para acceder a los
repositorios a través del proxy. ¿Cómo añadimos un nuevo repositorio?


\subsection{Ejercicio 5}
¿Qué diferencia hay entre telnet y ssh?


\subsection{Ejercicio 6}
Modifique la configuración de SSH para que impida el acceso remoto del
usuario root y cambie el puerto por defecto. Indique las líneas modificadas en el
fichero de configuración y ponga de manifiesto el cambio mediante capturas de
pantallas en las que se aprecie el comportamiento antes y después de los cambios.
Tenga en cuenta que debe reiniciar el servicio para que tome los cambios.


\subsection{Ejercicio 7}
Configure una instancia de Linux de forma que pueda acceder
remotamente (desde otra instancia o desde su anfitrión) sin introducir contraseña
(Pistas: ssh-keygen, ssh-copy-id). Documente el proceso que ha seguido indicando y
explicando los comandos utilizados así como posibles cambios en la configuración del
servicio. Muestre con capturas de pantalla que puede conectar al servidor ssh remoto
sin introducir contraseña.


\subsection{Ejercicio 8}
En muchas ocasiones es necesario reiniciar un servicio para que tome los
cambios en su configuración. Indique los comandos que puede emplear en Ubuntu y
CentOS para hacerlo.


\subsection{Ejercicio 9}
Ponga de manifiesto el funcionamiento de PHP en Apache creando un
fichero php que presente su nombre y apellidos y accediéndolo con un navegador web
Presente la captura de pantalla del resultado. Ponga de manifiesto el funcionamiento
de MySQL accediendo a la BBDDs por defecto (mysql) y consultando los usuarios
definidos en el sistema (select * from user). Documente con capturas de pantalla el
acceso y resultado de la consulta.


\subsection{Ejercicio 10}
Para poner de manifiesto que el servidor está funcionando, acceda con
un navegador web a su propio equipo (localhost). Cree una página HTML básica con su
nombre y apellidos y publíquela en su servidor IIS. Muestre, con una captura de
pantalla, como accede a dicha página con el navegador web.


\subsection{Ejercicio 11}
Escriba un breve contenido en un fichero de texto plano, cópielo y
modifíquelo ligeramente en un segundo archivo, por ejemplo, añadiendo un par de
líneas. Calcule las diferencias entre el fichero original y el modificado. Indique los
comandos necesarios para aplicar el parche así generado sobre el primer archivo y
obtener el segundo. Documente el proceso con capturas de pantalla de cada paso.


\subsection{Ejercicio 12}
Realice la instalación de esta aplicación y pruebe a modificar algún
parámetro de algún servicio. Muestre las capturas de el proceso de modificación y
ponga de manifiesto el resultado.



\end{document}